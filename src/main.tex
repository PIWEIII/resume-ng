% !TeX TS-program = xelatex

\documentclass{resume}
\ResumeName{刘生伟}

% 如果想插入照片,请使用以下两个库。
% \usepackage{graphicx}
% \usepackage{tikz}

\begin{document}

\ResumeContacts{
  (+86)130-1696-2873,%
  \ResumeUrl{mailto:markryu1019@gmail.com}{markryu1019@gmail.com},%
  % \ResumeUrl{https://blog.fkynjyq.com}{blog.fkynjyq.com} \footnote{下划线内容包含超链接。},%
  % \ResumeUrl{https://github.com/fky2015}{github.com/fky2015}%
}

% 如果想插入照片,请取消此代码的注释。
% 但是默认不推荐插入照片,因为这不是简历的重点。
% 如果默认的照片插入格式不能满足你的需求,你可以尝试调整照片的大小,或者使用其他的插入照片的方法。
% 不然,也可以先渲染 PDF 简历,然后用其他工具在 PDF 上叠加照片。
% \begin{tikzpicture}[remember picture, overlay]
%   \node [anchor=north east, inner sep=1cm]  at (current page.north east) 
%      {\includegraphics[width=2cm]{../resource/image.png}};
% \end{tikzpicture}

\ResumeTitle

\section{工作经历}
\ResumeItem
[公司|职位|最好一次绩效]
{贝壳找房(北京)科技有限公司}
[\textnormal{高级工程师(P6)|\textbf{ 2024年绩效5(杰出)}}]
[2022.03—至今]
\begin{itemize}
  \item \textbf{独立负责实时/离线标签系统建设,}构建Lambda架构支撑全人力方向27种标签数据分发,冷热数据计算分离、逻辑统一。
  \item \textbf{核心人事系统去PS化改造,}逻辑开关与接口DIFF实现服务平滑切换,重构人力时间轴读写逻辑,实现贝壳人力系统全面自研化。
  \item \textbf{AI赋能招聘业务流程,}OCR+LLM简化复杂业务表单的填写,RAG+LLM实现内部活水职位智能推荐。
  \item \textbf{人力中台核心研发,}优化、迭代贝壳P0级B端服务,迭代0故障,稳定性99.99\%,
  \item \textbf{构建人事报表平台与数据一致性体系,}高性能报表数据产出与读取,通过离线数据任务实现跨系统数据对齐率 99.91\%。
  \item \textbf{带领企业文化方向研发团队(4人),}负责需求拆解、方案评审与交付质量把控,保障多系统稳定迭代。
\end{itemize}

\section{项目经历}

\ResumeItem{\textbf{标签系统}}
[ \textnormal{Lambda架构| HiveSQL→Java(DFS) | LLM \& Prompt | 策略模式}]
\begin{itemize}
  \item 构建Lambda架构实现实时/离线标签计算分发,T-1数据以HiveSQL/Spark任务计算,实时数据通过Kafka处理,StarRocks为两类数据提供统一查询支持。
  \item 自研HiveSQL→Java解析器,将业务规则编码嵌入Kafka计算任务,实现计算逻辑统一。
  \item 设计标签快照转时间轴算法,基于窗口与偏移计算,实现历史标签高效追溯与查询,使得MySQL可支撑上亿级标签数据的高效读取。
  \item 对规则代码进行Druid语法树遍历分析,简化规则代码复杂度,配合LLM、Prompt将简化的代码转为自然语言,统一业务语言。
  \item 策略模式实现基于Apollo热配置的通用刷新模块,解决对象类型配置热更新问题。
  \item 完整实现Spring Cache抽象,支持Redis分布式缓存方案,提升标签接口查询性能。
\end{itemize}

\ResumeItem{\textbf{招聘 \& AI}}
[ \textnormal{LLM \& Prompt | OCR | RAG | 微服务架构 | Redisson}]
\begin{itemize}
  \item RAG + LLM 实现智能职位推荐,基于岗位职责与任职要求自动生成岗位标签,结合求职者简历与行为数据,实现个性化岗位推荐。
  \item OCR + LLM 实现招聘表单智能填写,基于OCR技术提取简历关键信息,结合 LLM 技术生成表单内容,提升招聘效率。
  \item 微服务架构设计,LLM交互模块独立部署,支持多模型接入与切换,提供统一API接口。
  \item Redisson分布式锁解决复杂表单暂存数据并发写入问题,保障数据一致性。
\end{itemize}

\ResumeItem{\textbf{人力中台 \& 核心人事系统(KeHR、LinkHR)}}
[ \textnormal{微服务 | 任务调度 | 最终一致性 | 缓存}]
\begin{itemize}
  \item Reactor响应式编程模型,优化核心员工数据读取接口(平均qps:1800),P99:300ms → 50ms,提升服务整体稳定性。
  \item 基于Redisson重构原分布式锁方案,解决并发场景下多租户数据同步覆盖问题,保障数据一致性。
  \item 生产-消费模型优化员工报表数据产出任务,多进程任务简化为单进程多线程任务,降低任务复杂度的同时提升数据产出效率(83%)。
  \item 结合定时任务与Apollo配置,实现通用外部系统数据推送模块,向北森、帆软等标准系统推送员工与组织架构数据,极大降低数据对接成本。
\end{itemize}

\section{教育经历}
\ResumeItem
[河海大学|本科生]
{河海大学}
[\textnormal{计算机科学与技术,计算机科学与技术学院|} 工学学士]
[2018.09—2022.06]

\textbf{GPA: 4.5/5.0(专业前 20\%)},获专业奖学金多次。

\section[技术能力]{技术能力}
\begin{itemize}
  \item \textbf{语言}: 掌握Java编程技术。常用 Python, Golang;熟悉Vibe Coding故编程不受特定语言限制。可使用\GrayText{TypeScript}、\GrayText{shell}。
  \item \textbf{存储}: 熟悉MySQL、Redis、ElasticSearch,解决业务场景问题。熟练使用HiveSQL进行离线数据处理。
  \item \textbf{消息队列}: 熟悉Kafka,理解其原理及应用场景。
  \item \textbf{工作流}: Git, GitHub, GitLab,ChatGPT,Copoilot。
  \item \textbf{其他}: 有容器化技术的实践经验,熟悉CI/CD(编译→镜像→部署)。
\end{itemize}

\section{个人总结}

\begin{itemize}
  \item 本人乐观开朗、在校成绩优异、自驱能力强,具有良好的沟通能力和团队合作精神。
  \item 可以使用英语进行工作交流(六级成绩 XXX),平时有阅读英文书籍和口语练习的习惯。
  \item 有六年 Linux 使用经验,较为丰富的软件开发经验、开源项目贡献和维护经验。善于技术写作,持续关注互联网技术发展。
\end{itemize}


\end{document}
