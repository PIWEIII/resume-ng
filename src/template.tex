% !TeX TS-program = xelatex

\documentclass{resume}
\ResumeName{刘生伟}

% 如果想插入照片,请使用以下两个库。
% \usepackage{graphicx}
% \usepackage{tikz}

\begin{document}

\ResumeContacts{
  (+86)130-1696-2873,%
  \ResumeUrl{mailto:lsw1019@foxmail.com}{lsw1019@foxmail.com},%
  3.5年经验,%
  1999-10,%
  % \ResumeUrl{mailto:markryu1019@gmail.com}{markryu1019@gmail.com},%
  % \ResumeUrl{https://blog.fkynjyq.com}{blog.fkynjyq.com} \footnote{下划线内容包含超链接。},%
  % \ResumeUrl{https://github.com/fky2015}{github.com/fky2015}%
}

% 如果想插入照片,请取消此代码的注释。
% 但是默认不推荐插入照片,因为这不是简历的重点。
% 如果默认的照片插入格式不能满足你的需求,你可以尝试调整照片的大小,或者使用其他的插入照片的方法。
% 不然,也可以先渲染 PDF 简历,然后用其他工具在 PDF 上叠加照片。
% \begin{tikzpicture}[remember picture, overlay]
%   \node [anchor=north east, inner sep=1cm]  at (current page.north east) 
%      {\includegraphics[width=2cm]{../resource/image.png}};
% \end{tikzpicture}

\ResumeTitle

\section{工作经历}
\ResumeItem
[公司|职位|最好一次绩效]
{贝壳找房(北京)科技有限公司}
[\textnormal{高级工程师(P6)|\textbf{ 2024年绩效5(杰出)}}]
[2022.03—至今]
\begin{itemize}
  \item \textbf{核心职责:}负责贝壳内部人力系统(KeHR、LinkHR、招聘)的运维与迭代,支撑17万+员工的人力业务需求。负责迭代、优化核心B端系统人力中台,支撑全公司业务系统的人力数据需求。独立设计并落地人力标签系统,支撑全公司多业务线标签计算与获取需求。完成集团人力OLAP场景的数据建模,构建OLTP→OLAP分析场景。
  \item \textbf{工程能力:}良好的PRD→功能转化能力、Lambda架构的设计与实现、高并发接口优化经验、分布式系统开发、AI(LLM)应用技术开发、Hive离线数据处理能力。
  \item \textbf{领导力:}带领企业文化方向研发团队(4人),负责需求拆解、方案评审与交付质量把控,组织CodeReview,保障多系统稳定迭代。
\end{itemize}

\section{项目经历}

\ResumeItem{\textbf{标签系统}}
[ \textnormal{Lambda架构| HiveSQL→Java(DFS) | LLM \& Prompt | 策略模式}]
\begin{itemize}
  \item 构建Lambda架构落地实时/离线标签计算分发,T-1数据以HiveSQL/Spark任务计算,实时数据通过API/Kafka处理,StarRocks为两类数据提供统一查询支持。
  \item 独立研发HiveSQL→Java解析器,将业务规则编码嵌入API/Kafka计算任务,实现计算逻辑统一。
  \item 设计快照类型数据转时间轴算法(基于开窗函数与偏移量计算)实现历史标签高效追溯与查询,使得MySQL可支撑上亿级标签数据的高效读取。
  \item 对规则代码进行Druid语法树遍历分析,简化规则代码复杂度,配合LLM、Prompt将简化的代码转为自然语言,统一业务语言。
  \item 策略模式实现基于Apollo热配置的通用刷新模块,解决对象类型配置热更新问题。
  \item 完整实现Spring Cache抽象,支持Redis分布式缓存方案,提升标签接口查询性能。
\end{itemize}

\ResumeItem{\textbf{人力中台 \& 核心人事系统(KeHR、LinkHR)}}
[ \textnormal{微服务 | 任务调度 | 最终一致性 | 缓存 | Redisson}]
\begin{itemize}
  \item Reactor响应式编程模型,优化核心员工数据读取接口(平均qps:1800),P99:300ms → 50ms,提升服务整体稳定性。
  \item 基于Redisson重构原系统的分布式锁方案,解决并发场景下多租户数据同步覆盖问题,保障数据一致性。
  \item 生产-消费模型优化员工报表数据产出任务,多进程任务简化为单进程多线程任务,降低任务复杂度的同时提升83\%的数据产出效率。
  \item 结合大数据离线任务 + MySQL实时数据,实现人力组织树数据的高性能查询接口,解决时间维度下组织架构需要递归查询的时间问题,P99:100ms。
  \item 针对人事领域人员、组织、岗位数据进行标准建模,HiveSQL任务配合StarRocks查询引擎,配合帆软落地高性能OLAP分析场景。
\end{itemize}

\ResumeItem{\textbf{招聘 \& AI}}
[ \textnormal{LLM \& Prompt | OCR | RAG | 微服务 | Redisson}]
\begin{itemize}
  \item RAG + LLM 实现智能职位推荐,基于向量检索技术从职位库中筛选相似职位,结合LLM技术生成推荐理由并计算匹配度。
  \item OCR + LLM 实现招聘表单智能填写,基于OCR技术提取简历关键信息,利用Prompt约束LLM输入输出,结合Java工程对LLM输出进行兜底,一键填写页面复杂表单,提升招聘效率。
  \item 微服务架构设计,与LLM交互模块独立部署,支持多模型接入与切换,提供标准API/SDK。
  \item 乐观锁解决复杂表单暂存数据并发写入问题,保障数据一致性。
\end{itemize}

\ResumeItem{\textbf{企业文化方向微服务群}}
[ \textnormal{ElasticSearch | Kafka | 微服务 | CI/CD}]
\begin{itemize}
  \item 利用ElasticSearch构建通用的企业文化系统后台列表分页查询方案,支持多维、动态权限过滤,提升查询性能与用户体验。
  \item 通过Kafka消息队列监听人力中台变更事件,确保企业文化系统数据数据权限与组织架构的绑定实时同步。
  \item 结合公司技术栈,为VibeCoding产物提供标准脚手架支持,设计标准化的前后端CI/CD流程,2月内帮助非技术人员快速上线轻应用3个。
\end{itemize}

\newpage
\section{教育经历}
\ResumeItem
[河海大学|本科生]
{河海大学}
[\textnormal{计算机科学与技术,计算机科学与技术学院|} 工学学士]
[2018.09—2022.06]
\begin{itemize}
  \item \textbf{GPA: 4.5/5.0(专业前 20\%)}: 获专业奖学金多次。
  \item \textbf{荣誉}: 服务外包创新创业大赛东部赛区三等奖;美国大学生数学建模S奖;2018年、2020年校网络文化艺术节网页设计大赛三等奖、二等奖。
  \item \textbf{专利}: 一种金融证券舆情信息爬取方法及装置、一种基于社交大数据的基金推荐系统及方法。
\end{itemize}

\section[技术能力]{技术能力}
\begin{itemize}
  \item \textbf{语言}: 掌握Java编程技术。常用 Python, Golang;熟悉Vibe Coding故编程不受特定语言限制。可使用\GrayText{TypeScript}、\GrayText{shell}。
  \item \textbf{语言特性}: 熟悉Java集合、JUC并发编程、JVM、垃圾回收。
  \item \textbf{框架}:熟练使用Spring、SpringBoot、SpringCloud等,了解其设计思想与部分实现。
  \item \textbf{存储}: 熟练使用MySQL、Redis、ElasticSearch去解决业务场景问题。熟练使用HiveSQL进行离线数据处理。
  \item \textbf{消息队列}: 熟悉Kafka,理解其原理及应用场景。
  \item \textbf{工作流}: Git, GitHub, GitLab,ChatGPT,Claude Code。
  \item \textbf{计算机科学与技术}:算法与数据结构、操作系统、计算机网络、数据库储存原理等。
  \item \textbf{其他}: 有容器化技术的实践经验,熟悉CI/CD(编译→镜像→部署)。
\end{itemize}

\section{个人总结}

\begin{itemize}
  \item 热爱Coding,享受知识落地的过程。
  \item 工作方面具有责任心与Owner意识,喜欢迎接挑战。
  \item 学习方面拥抱新技术,善于总结与分享。
  \item 可以使用英语进行工作交流(CET6 + 英语双学位),热爱进行英文练习。
\end{itemize}


\end{document}
